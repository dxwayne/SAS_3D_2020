\documentclass[letter,11pt,oneside]{article}

%%% (occur "\\(\\\\[a-z]*section\\|appendix\\|input\\|\\<include\\>\\)")

%%\documentclass[11pt,twocolumn]{article}
%%\usepackage[inline]{asymptote}   %% Inline asymptote diagrams
%%\usepackage{wglatex}             %% Use this one and kill others.
\usepackage{color}               %% colored letters {\color{red}{{text}}
\usepackage{fancyhdr}            %% headers/footers
%%\usepackage{fancyvrb}            %% headers/footers
\usepackage{datetime}            %% pick up tex date time 
\usepackage{lastpage}            %% support page of ...lastpage
\usepackage{times}               %% native times roman fonts
\usepackage{textcomp}            %% trademark
\usepackage{amssymb,amsmath}     %% greek alphabet
\usepackage{parskip}             %% blank lines between paragraphs, no indent
\usepackage{shortvrb}            %% short verb use for tables
\usepackage{lscape}              %% landscape for tables.
\usepackage{longtable}           %% permit tables to span pages wg-longtable
\usepackage{url}                 %% Make URLs uniform and links in PDFs
\usepackage{enumerate}           %% Allow letters/decorations for enumerations
\usepackage{endnotes}            %% Enhance footnotes/endnotes
\usepackage{listings}            %% Make URLs uniform and links in PDFs
\pdfadjustspacing=1                %% force LaTeX-like character spacing
\usepackage{geometry}            %% allow margins to be relaxed
%%\usepackage{wrapfig}             %% permit wrapping figures.
%%\usepackage{subfigure}              %% images side by side.
\geometry{margin=1in}            %% Allow narrower margins etc.
\usepackage[T1]{fontenc}         %% Better Verbatim Font.
\renewcommand*\ttdefault{txtt}   %% 
\usepackage[bookmarks]{hyperref} %% Make huperlinks within a PDF
%%\usepackage{natbib}   %% bibitems

%% include background image (wg-document-page-background) 

\usepackage{graphicx}            %% Include pictures into a document
%% (wg-texdoc-inserttikz)


\def\documentisdraft{DRAFT}  % NOTDRAFT

%% (wg-texdoc-isdraft)


\def\drafttest{DRAFT}
\def\wgdocdate{09 May, 2020}
\def\wgdocdatetime{\wgdocdate at \currenttime}
\ifx\documentisdraft\drafttest
\usepackage[left]{lineno}   %%%%%%%%%%%%% DRAFT
\usepackage{draftwatermark}
%%\SetWatermarkScale{5.0}
%%\SetWatermarkColor[gray]{0.3}
\fi

%% (wg-texdoc-insert-fancy-headers)

\usepackage[bookmarks]{hyperref} %% Make huperlinks within a PDF
%%\usepackage{makeidx}             %% Make an index uncomment following line
%%\makeindex                       %%.. yeah this one, too. index{key} in text
%%



\definecolor{verbcolor}{rgb}{0.6,0,0}
\definecolor{darkgreen}{rgb}{0,0.4,0}
\newcommand\debate[1]{\textcolor{darkgreen}{DEBATE: #1} \marginpar{\textcolor{red}{DEBATE} }}
\newcommand{\ltodo}[2]{\marginpar{\textcolor{red}{ACTION: #1}\endnote{#2}}}
\renewcommand{\thefigure}{\thesection-\arabic{figure}}
\newcommand{\menu}{\ensuremath{\;\rightarrow\;}}
\newcommand{\dhl}[1]{{\color{verbcolor}{\texttt#1}}}
\definecolor{wglightgreen}{rgb}{0.88, 0.58, 0.88}
\newcommand{\wgtextbox}[1]{\noindent\fcolorbox{darkgreen}{wglightgreen}{%
    \minipage[t]{\dimexpr0.80\linewidth-2\fboxsep-2\fboxrule\relax}
        {#1}
        \endminipage}}


\newcommand{\newlowspec}{Tenative Lowspec 4}
\newcommand{\transformer}{TODO Transformer}
\newcommand{\github}{\url{https://github.com/dxwayne/SAS_3D_2020}}
\newcommand{\readthedocs}{\url{https://SAS_3D_2020.readthedocs.io}}


%%(wg-add-inline-images)  %% add inline images to the mix


%%Begin User Definitions: Hint: ~/.latex.defs and  latex.defs  
%%End User Definitions:
%%(wg-texdoc-adjust-paper-width)
%% (wg-texdoc-insert-hypersetup)



%%%%%%%%%%%%%%%%%%%%%%%%%%%%%%%%%%%%%%%%%%%%%%%%%%%%%%%%%%%%%%%%%%%%%%%%%%%%%


\begin{document}

%% (wg-latex-pretty-title-page)
%% (wg-texdoc-titleblock)


\title{The Role of 3D Printing in Spectrograph and Small Telescope Science}
\author{Robert Buchheim, Jerry Foote, Wayne Green, Paul Gerlach, \\
  Anthony Rodda, Forrest Sims, Thomas Smith, Clarke Yeager}
\date{31 May, 2020}
\maketitle

\begin{abstract}
   Anthony Roddy, following Paul Gerlach's LOWSPEC design, introduced
   this team to the power of 3D printing for astronomical purposes at
   the AAVSO Fall Conference in 2018. This was reinforced at the
   Sacramento Mountain Spectroscopy Workshop 2 (SMSW-2) in Las Cruses,
   NM in February. This repository contains results, videos and other
   resources in support of 3D spectroscopy. The poster presents a host
   of considerations about calibration lamps, thermal expansion issues
   related to ABS,PETG and other 3D media, optics, automation,
   telemetry, and adapting to other focal-lengths. This paper
   introduces a repository at \github. This paper introduces 3D
   printing for other areas of small telescope science.
\end{abstract}

%%%%%%%%%%%%%%%%%%%%%%%%%%%%%%%%%%%%%%%%%%%%%%%%%%%%%%%%%%%%%%%%%%%%%%%%%%%%%
%% table of contents
%%%%%%%%%%%%%%%%%%%%%%%%%%%%%%%%%%%%%%%%%%%%%%%%%%%%%%%%%%%%%%%%%%%%%%%%%%%%%
\ifx\documentisdraft\drafttest
\renewcommand*\contentsname{REMOVE CONTENTS FROM PAPER}
\pagenumbering{roman}   % i,ii,etc
%%\pagenumbering{gobble}   %ignore page numbers for a while
\pdfbookmark[0]{Table of Contents}{MyTOC} % if usepackage{hyperref} in use.
\tableofcontents
\listoffigures
\listoftables
\newpage
\fi


\setcounter{section}{0}
\pagenumbering{arabic}

\ifx\documentisdraft\drafttest
\linenumbers    %%%%%%%%%%%%% DRAFT
\fi

\section{Overview}

Foote \cite{Lowspec2} during the SAS 2019 Symposium presented the
results of an international collaboration of small telescope scientists
to replicate, explore and to adapt the Lowspec 2 project \cite{Gerlach}
for 3D printing R$\sim$1000 spectroraph. This paper presents details
that this collaboration has taken towards a new and improved \newlowspec
device.

Yeager discovered discrepancies with the Lowspec 2 under thermal loads
encountered during the winter conditions in Colorado, USA.  This lead
to using an Arduino and Maxim/Dallas 18B20 temperature sensors to
monitor differential temperatures while taking calibration
exposures. The spectrograph was not moved between exposures to rule
out flexure and other aspects. The conclusion is a shift in the
position of the calibration lines shift with respetc to emperature
over a duration matching a nominal exposure sequence.

He also experimented with introducing flexure into the instrument
by proping up a corner. This also demonstrated a shift  in the
position of the calibration lines shift with respect to orientation
is an issue.

See figures ...

The layout of the Lowspec is:

\ltodo{LS Layout}{Two figures LS2/3 with callouts for parts
  mentioned and arrows for differences.}

Other general issues include the orientation of the spectrograph
to correct for parallactic angle, guiding, using more than two
cameras at the same time with one computer, the cabling associated
with the instruments, length of the cables required for larger
diameter OTAs, several aspects of focusing the instrument,

...


\section{Issues Addressed}

\subsection{Calibration Lamp}

The group experimented with local Ne/Ar ballast lamps for
fluorescent lights. These are small easily acquired  modules
from the local home improvement stores. They consist of
a simple resistor in series with a sealed bulb. Neon only
lamps are available (replacements for indicator lamps on
older appliances like freezers, etc) from hardware stores.
These operate on 60 Hz 120 VAC mains in the United States
with a peak RMS of around 180 VAC. This is sufficient to
trigger these lamps. Most ballast bulbs have a built-in
bi-metallic strip that heats up and shorts across the filiments
causing light output to all but disappear. In Europe, 50Hz
250 VAC is the norm. The European Relco\texttrademark lamps
last longer on US mains, but the bi-metallic strip still
hampers direct mains to the payload operation.

Walker \cite{Walker-1} published a 555 timer / transformer
circuit using a German transformer. Green and Yeager located
a transform \transformer more readily available in the US and Yeager
adapted the circuit and produced a preliminary Printed Circuit
Board (PCB). The design runs the transformer backwards by
hooking 6-12 VAC to the output side causing a 250 VAC output.
The DC input is converted to a pulse modulated signal by
the 555 timer. The best operating frequency is around 500Hz.

\ltodo{555 Circuit}{Show waveforms, spectra etc.}

After this initial experiment, the group decided that the
real-estate on the PCB was low, and what other features
could be adapted to the box to support other peripheral
sensors.

Experiments were done along the lines of shape of the input beam
into the slit of the spectrograph. The premise assumed a
converging beam that more closely matched the Point Spread
Functon (PSF) of the instrument would be best. Foote and Smith
devised a moterized device to lower a mirror into place
and choose Ne/Ar or flat tungsten lamp light input.

Avaliablilty of calibration lamps other than Neon are
a problem. Argon based lamps are not as readily available
as they once were. 

\ltodo{Jerry's Decice}{Show diagrams etc of this.}

Rodda took the expediency of placing the bulb directly
adjacent to the Schmidt Cassegrain Telescopes's (SCT)
secondary. This eliminated the need for extra weight
etc on the payload.

\subsection{Moment of Inertia}

Most spectrographs are developed along the optical axis. This
results in placing the camera at a position farther from the
back of the OTA and poses a clearance issue with fork mounted
scopes. The lowspec is no issue.


\subsection{Thermal Shifts}

The southern part of the United States experiences large shifts
in diurnal temperature. The expansion coefficent of the printing
plastics are on the order of 50 microns per meter per K.
The coefficents are on the order of 21 and 9 $\mu{m}$/m/K
respectivly. The current layout of the Lowspecs 2 and 3
have a larger solid block to hold the guide camera. This
block comes to equilibrium slower than the rest of the box
and introduces torque on the optical axis. Other observers,
located in Europe for example do not experience these fast
temperature swings.

\subsection{Focus}

Focusing the spectrograph's collimination lens, the camera lens
the placement of the camera onto the body, and the orientation
of the camera is an issue. The weight of the camera was too high
to use printed threads so metalic camera adapters were embedded
into the body.


\subsection{Guiding}

Placement, orientation, field of view and focus of the guide
camera has been re-designed. A LED was added inside the
spectrograph to momentarily illuminate the slit to establish
its orientation w.r.t. the guider sensor.

A deep exposure is taken to establish the field, and plate-solved.
The plate solution can be achieved by visiting a field at the
same declination with sufficent stars to solve - and the
WCS coordinates can be altered and transferred to the new
field. Once distortoins are known, it is possible to synthesize
a astrometric WCS to reasobable precision.

The Field of View (FOV) of the guide camera has to be large
enough to allow guiding software to work but wide enough
to grab an off-slit star when seeing is particullary good.

\subsection{Gratings and Grating Orientation}

The working dispersion range is controlled by the focal length
of the ``camera'' lens as well as the number of lines per
mm of the grating. Orientation of the grating is critical
to avoid order-overlap. We developed software to assist
with the planning and layout of the optical components.
Anamorphic magnification is also a consideration.

\subsection{The Slit and Slit Assembly}

The slit assembly has been redesigned several times. The new
approach will employ a motor to select the slit at the scope.
New slits are being considered as well.

\section{Camera and Instrument Control}

John Hoot \cite{} introduced the idea of using a ODroid XU4 Single
Board Computer (SBC) using an eMMC flash drive and running an ARMxxx
Ubuntu 18.04. The heart of the main software suite includes KStars,
Ekos, and INDI capable of controlling 2 USB cameras.  The SBC is small
and light enough to be attached to the SBC.  The SBC requires only
Power, Ground and an Ethernet cable and is controlle through X-Windows
protocols. The software suite has been augmented by compiling ds9
\cite{} from source on the SBC, adding Sextractor \cite{} and
Astrometry.net \cite{} with a subset of the index files as required
for the field size. The ds9 program assists with focus.

Images are acquired and stored on the SBC, and may be simultaneously
written via SMB or NFS protocols to a remote computer. The
SBC also supports a distributed MQTT broker service and handles
telemetry reports from various additional instruments discussed
herein.


\ltodo{XU4 Processor}{Get the description of ODroid together}

\section{Telemetry}

The Arduino Nano 33 is a single board computer that supports a number
of sensors on the card. The main sensors are a microphone, a 9-axis
accellerometer, temperature sensors and light sensors.  The Arduino
computers use a subset of C++ and classes have been written to adapt
each sensor to the small telescope spectroscopy needs. Telemetry is
``published'' by the Arduino, and managed by a MQTT broker, here on
the ODroid. C++ and Python classes are available, see the electronic
appendices for this paper. \footnote{A public GitHub repository may be
  found at \github and documentation is
  at \readthedocs.}

MQTT \cite{} is a Publish/Subscribe protocol. Messages are simple,
but may include embedded text-only JSON structures. Each sensor
may publish its data and many subscribers may receive the messages.
The sensor module may subscribe to a topic as well and recieve
instructions. A non-MQTT out-of-band serial control structure has
been developed to allow one or more Arduino devices to take sensor
readings, perform basic actions etc in a distributed manner.

Overall, the MQTT is well known and well supported. Control can
be by special code modules or via the Node-RED \cite{} web-based
mechanism and a simple browser.




\section{New Directions}

The main box is being replaced with a commercially available
metal box. This requires some machining. Clever hand tools
and additional printing can be employed. Several of the authors
have small machine shops at their disposal. The optical layout
is achieved with printed parts to hold each of the components.

...




\section{Conclusions}

The new design will address the moment arm issue, 



%%\appendix
%%\renewcommand \thesection{\Alph{section}}

%% use a bibitem approach to the references publications etc.
%% (wg-bibitem)

%%\clearpage
%%\addcontentsline{toc}{section}{References}
%%\renewcommand*{\refname}{My Bibliography and References}
%%\bibliographystyle{plain}	% bibliographystyle{apalike} and \usepackage{natbib}
%%\bibliography{MasterBib}	% expects file "MasterBib.bib"



%%\begin{thebibliography}{80}
%%\usepackage{natbib}   %% bibitems
%%\end{thebibliography}

%%\clearpage
%%\addcontentsline{toc}{section}{Index}
%%\printindex %% www.cs.usask.ca/resources/tutorials/latex/notes/toc-index.pdf

% /home/git/pre/Lowspec3.pre/doc/Lowspec3.tex

%% (wg-texdoc-endnotes)

%%%%%%%%%%%%%%%%%%%%%%%%%%%%%%%%%%%%%%%%%%%%%%%%%%%%%%%%%%%%%%%%%%%%%%%%%%%%%
% Support for endnotes
\begingroup
\renewcommand{\notesname}{\textcolor{red} {Action Items:}}
\parindent 0pt
\parskip 2ex
\phantomsection
\addcontentsline{toc}{section}{	extcolor{red} {Action Items:}}
\def\enotesize{\normalsize}
\theendnotes
\endgroup

\end{document}
